\cleardoublepage
\chapternonum{攻读博士学位期间取得的科研成果}
\sectionnonum{攻读博士学位期间发表的论文}

\begin{enumerate}
    \item 第一作者. Towards Secure and Practical Machine Learning via Secret Sharing and Random Permutation. Knowledge-based Systems. 2022. (SCI一区,对应第5章)
    %
    \item 第一作者. Reducing Communication for Split Learning by Randomized Top-$k$ Sparsification. Proceedings of the Thirty-Second International Joint Conference on Artificial Intelligence. 2023. (EI, CCF-A,对应第3章)
    %
    \item 第一作者. Protecting Split Learing by Potential Energy Loss. Proceedings of the Thirty-Third International Joint Conference on Artificial Intelligence. 2024. (EI, CCF-A,对应第4章)
    %
    \item 学生一作. Survey and Open Problems in Privacy-Preserving Knowledge Graph: Merging, Query, Representation, Completion, and Applications. International Journal of Machine Learning and Cybernetics. 2024. (SCI)
    %
    \item 学生二作. Federated Learning on Non-iid Data via Local and Global Distillation. IEEE International Conference on Web Services (ICWS). 2023. (EI, CCF-B)
\end{enumerate}

\sectionnonum{攻读博士学位期间申请的专利}
\begin{enumerate}
    \item 第二发明人. 一种基于共享学习的神经网络模型的训练方法. \\
    (已授权,申请公布号 CN112183730A)
    %
    \item 第三发明人. 基于噪音蒸馏的联邦学习系统及方法. \\
    (已授权,申请公布号
     CN114819196A)
    %
    \item 第四发明人. 针对标签推理攻击的联邦学习模型训练方法. \\
    (实质审查,申请公布号 CN117610676A)
\end{enumerate}



\sectionnonum{攻读博士学位期间参与的科研项目}
\begin{enumerate}
    \item 国家重点研发计划项目“大数据征信及智能评估技术”\\
    (编号 2018YFB1403001)(2019 年7月至2022年6月)
    \item 国家自然科学基金管理学部重大项目“生产和服务管理决策中的人机协同新型模式研究”\\
    (编号 72192823)(2022年1月至2026年12月)
\end{enumerate}