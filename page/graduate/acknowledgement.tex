\cleardoublepage
\chapternonum{致谢}
在这气温超过40度的酷暑时节,我终于完成了博士毕业论文的写作,标志着五年的博士生涯进入了尾声。
%
回想起五年前初次踏入玉泉大门的时刻,仿佛仍在昨日。
%
在五年的博士生涯中,各位老师、同学,以及我的家人朋友们给予了我莫大的帮助,我在此深表感谢。

首页我要感谢我的导师郑小林教授,郑老师平易近人,治学严谨,在学生遇到困难时总会及时帮助,在我科研遇到挑战时给予了我许多关照。
%
郑老师也让我有幸参与了许多重大的科研和业界项目,让我接触到了各方面的先进知识和信息,让我对学术界、产业界有了更加深刻的理解,受益匪浅。

其次我要感谢实验室的陈超超研究员,陈老师就像个同龄人一般与我们进行交流和讨论,在学术上给予了我细心的指导和帮助,在我投稿论文时进行了逐字逐句的指导,也告诉了我许多关于科研、工作的经验,让我能够更加沉稳平和地面对科研中的困难。

此外,我还要感谢实验室的朱梦莹老师,也是我曾经的师姐。
%
朱老师以严格的要求让我们高标准地完成了实验室的基金申请等许多任务,并且乐于分享,让我们更加了解最新的学术界、产业界的风向。

我也十分感谢实验室的同学们,包括师兄李其柄、李宇渊,师姐檀彦超、梁倩乔,
%
我的同门应森辞、吴锐、魏翔宇,
%
师弟韩钟萱、周芃旸、求昊泽,
%
师妹廖馨婷,
%
以及实验室的其他同学,包括张海宁老师、刘俊麟、周千遇、张胜嘉、马定伟、成文杰、王逸豪、张亦钊等学弟学妹。
我们在实验室一同成长,一起面对各种艰难的项目,一起吃喝玩乐,度过了许多欢乐时光。
%
我也要感谢我的舍友辑凯文、李植、赵海亮,以及杨博、房聪、田老师、印阳等同学朋友,
你们在各个方面给了我许多帮助,也让我的研究生生活更加丰富多彩。

最后,我感谢我的父母,感谢你们20多年来的照顾、陪伴和支持,让我能够安心读书而无需担心过多。你们是我人生的最大支柱。

\vspace{20pt}
\rightline{郑非\phantom{2024}}
\rightline{2024年9月}
