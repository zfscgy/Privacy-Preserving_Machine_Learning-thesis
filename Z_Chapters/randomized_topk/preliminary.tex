\section{压缩方法初步研究}
本节介绍了几种基础的可以对拆分学习的通讯量进行压缩的方法,包括缩小拆分层、top-$k$稀疏、量化拆分层、使用$L1$损失函数诱导稀疏,并对各方法进行初步分析。
%
在以下的分析中,我们将拆分学习的底部模型记作$M_b$,顶部模型记作$M_t$,输入、中间结果、输出分别记作$X, H, Y$。
我们使用$d$表示初始的拆分层的维度,$k$表示拆分层压缩后的维度。


\textbf{缩小拆分层}:
在拆分学习中,每一轮训练或推断中都会传输拆分层的值以及(训练时)梯度。
%
因此,缩小拆分层的大小可以直接减少拆分学习每一轮的通讯量。
%
假设1个全连接网络的隐层维度分别是1000(输入)-100(第1层)-100(第2层/拆分层)-10(输出层),则每一轮前向传播需要传输100个数值;若把拆分层维度缩小到10,则只需要传输10个数值。
%
同样地,在反向传播过程中,也只需要传输10个梯度值。
%
因此,这种方法可以削减90\%的每轮通讯量。

\textbf{量化拆分层}:
量化指的是将较高比特位的值(如32位浮点数)转化为较小比特位的近似值(如16位浮点数,4位整数,甚至1位的0-1布尔值)~\cite{zhou2016dorefa,banner2018_8bit,yang2019quantization}。
%
其在神经网络中被广泛应用于减少计算和存储开销。
%
通过对拆分层的值以及梯度进行量化压缩,也可以减少拆分学习训练和推断过程中的通讯量。
%
本文考虑最常用的等距均匀量化(Uniform Quantization),对于一个隐层向量$\bvec{h} = (h_1, \cdots, h_d)$,将其量化为 $b$ 比特的量化方法为:
\begin{equation}
    \bvec{h}^C = \mathsf{Compress}(\bvec{h}) = \left( \left\lfloor \dfrac{h_1 - h_\text{min}}{h_\text{max} - h_\text{min}} \cdot  2^b \rfloor, \cdots, \lfloor \dfrac{h_n - h_\text{min}}{h_\text{max} - h_\text{min}} \cdot 2^b \right\rfloor \right),
\end{equation}
其中,$h_\text{min}$ 和 $h_\text{max}$ 分别表示向量$\bvec{h}$中的最小值和最大值。
%
而解压缩的过程则可以表示为:
\begin{equation}
    \bvec{h}' = \mathsf{Decompress}(\bvec{h}^C) = \left(\cdots, h_\text{min} + (h^C_i + \dfrac12)(\dfrac{h_\text{max} - h_\text{min}}{2^b}), \cdots \right).
\end{equation}
% 
尽管量化压缩可以应用于前向传播的隐层值和反向传播的梯度值,但是在实验中我们发现,如果将量化压缩同时应用于前向传播和反向传播,会带来严重的模型效果损失。
%
考虑到本文主要关注拆分学习推断过程中的通讯压缩,因此只将量化压缩应用于反向传播的隐层梯度中。

\textbf{Top-$k$稀疏化}:
Top-$k$稀疏化指在向量中只保留$k$个绝对值最大的元素,而将其他元素设为0.
%
在这种情况下,只需要传输$k$个元素的值和下标。
%
如果在前向传播中应用了top-$k$稀疏化,则较小的$d-k$个元素没有参与顶部模型的计算,因此,在反向传播过程中也无需传播这$d-k$个元素的信息。
%
因此,top-$k$稀疏化可以自然地同时应用于前向传播和反向传播。
%

\textbf{L1正则化}:$L1$正则化通过$L1$损失来诱导稀疏性,被广泛应用在不同的机器学习领域~\cite{tibshirani1996lasso,wright2008sparse_face,yin2012kernel,hoefler2021sparse_deeplearning}。
%
具体来说,在拆分学习的训练过程中,我们在原有的损失函数上加入一个对于拆分层的隐层向量的$L1$损失:$L' = L + \lambda \sum_{i=1}^d |h_i|$。
%
其中,$\lambda$ 用于控制稀疏化的强度,大的$\lambda$会导致更高的稀疏性,但是同时也会降低模型训练的效果。
%
由于$L1$正则化需要在训练过程中逐渐实现稀疏化,因此其稀疏化只适用于模型的推断阶段,此时我们可以将隐层中接近0的元素删除,类似top-$k$稀疏化来传播绝对值较大的值和对应的下标。
%


我们将各种方法对应的压缩比率(压缩后大小/压缩前大小)总结在表 \ref{tab:randomized_top-k:compression_ratio}中。
%
对于top-$k$即$L1$稀疏化,我们未考虑对下标进行压缩。
%
表中 $N$ 表示初始值的比特位数,一般为32,$k$表示保留的元素的个数。

\begin{table}[h]
    \centering
    \begin{tabular}{ccc}
    \toprule
    \multirow{2}{*}{压缩方法} & \multicolumn{2}{c}{压缩比率} \\ 
    \cmidrule(lr){2-3}
            & 前向传播 & 反向传播 \\ \midrule
    拆分层缩小     & $k/d$       & $k/d$ \\
    $b$比特量化    & $2^b/N$    & 1 \\
    Top-$k$稀疏化  & $k/d\cdot (1 + \lceil \log_2 d \rceil/N)$ & $k/d$ \\
    L1正则化       & $k/d\cdot (1 + \lceil \log_2 d \rceil/N)$ & 1 \\
    \bottomrule
    \end{tabular}
    \caption{不同压缩方法的压缩比率}
    \label{tab:randomized_top-k:compression_ratio}
\end{table}