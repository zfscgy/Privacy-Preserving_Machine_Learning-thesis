本文对应的研究有如下创新点:
\begin{enumerate}
    \item 提出随机top-$k$稀疏化优化了拆分学习的通讯压缩:针对现有的拆分学习通讯压缩研究缺少的问题,对常用压缩算法在拆分学习中的应用进行了研究,并且通过理论分析提出了随机top-$k$稀疏算法,实现了通讯高效的拆分学习训练和推断。
    %
    \item 提出势能损失提高了拆分学习的隐私水平:针对拆分学习中的模型补全攻击,提出从泛化误差视角看待隐私泄漏问题,受物理学现象启发提出了势能损失函数,对拆分学习中底部模型带来的隐私泄露进行了防护。
    %
    \item 将随机排列与秘密分享结合:针对纯密码学的隐私保护机器学习框架效率慢、精度损失、实现困难的问题,从非线性激活函数着手,提出了采用随机排列辅助非线性激活函数的计算,在尽可能保护安全性的同时极大提高了效率,实现了隐私保护神经网络的训练和推断。
    %
    \item 基于随机排列和密码学技术,对大语言模型实现了秒级安全推断:针对大规模语言模型计算量、通讯量巨大、传统密码学方法难以实现隐私推断的问题,对秘密分享进行优化,并且结合安全随机排列协议、基于全同态加密的隐匿查询,实现了在常见网络环境下的秒级别大模型推断,比现有密码学方法提高两个以上数量级。
\end{enumerate}