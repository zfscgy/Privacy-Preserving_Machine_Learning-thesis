\chapter{相关研究综述}
隐私保护机器学习(Privacy-Preserving Machine Learning)指的是在保护数据以及模型隐私的前提下进行机器学习模型的训练和推断~\cite{tan_2020_PPML_survey_ch}。
%
当前的隐私保护机器学习包含了基于密码学的隐私计算、拆分学习、联邦学习、可信执行环境、差分隐私等多种方法。
%
基于密码学的隐私计算~\cite{al_2019_ppml}通过各种密码学底层技术实现可证明安全的隐私保护机器学习,但是由于牵涉到大量密码学计算,会带来较高的通信和计算开销。
%
拆分学习~\cite{vepakomma2018split}通过切分模型和交换中间结果实现隐私的训练和推断,具有实现简单、计算和通信开销小的优点,但是由于部分模型和中间结果的暴露,存在一定的隐私泄露风险。
%
联邦学习\cite{yangqiang2019federated,mcmahan_2017_fedavg, zhouchuanxin_2021_fed_survey_ch}通过多个参与方聚合模型来保护各个参与方的本地数据隐私,适用于数据横向分割、训练阶段的场景。
%
可信执行环境~\cite{sabt_2015_tee,2016_intel_sgx}将模型和数据装载入特定可以被信任的硬件中进行运算,与外部环境隔离,从而实现隐私保护。
%
差分隐私~\cite{dwork_2006_differential_privacy,wuruihan_2023_label_dp}通过在计算过程中加入特定分布的噪声,使得计算结果满足一定的隐私性,可以控制输入数据的隐私泄露程度,但是其必须在隐私保护程度和计算精确性之间进行权衡~\cite{abadi_2016_dp_dl}。
%
上述的几类方法彼此相对独立,但是又在一定程度上相互重叠,大量研究~\cite{zhangqiao_2018_gelu_net,bonawitz_2017_secure_agg,thapa_2022_splitfed,zhou_2022_codesign,riazi_2018_chameleon,weikang_2020_fed_dp}采用其中的多种方法来共同构建新的隐私保护机器学习方法,旨在更好地平衡效率和安全。
%
我们将上述的各种隐私保护机器学习的研究方向归纳在\autoref{tab:related_work}中。
%
本文主要研究通用的适用于多方场景的隐私保护机器学习算法框架,重点面向两方进行隐私保护的神经网络联合训练和推断的场景,因此聚焦于拆分学习、密码学方法以及密码学和非密码学的混合方法三个方面。
%
本章节对以上的三个方面的研究现状逐一进行更细致的介绍。

\begin{table}[h]
    \small
    \label{tab:related_work}
    \caption{隐私保护机器学习相关研究概览}
    \begin{tabular}{p{45pt}p{250pt}p{22pt}cp{22pt}p{22pt}}
    \toprule
        研究方向       & 描述                                                                                       & 适用范围 & 效率 & 安全性 & 模型精度 \\ \midrule
        基于密码学的隐私计算 &
          通过密码学协议,实现安全的模型训练和推断。适用于单次计算中参与方较少(2方或3方)的场景(包括纵向联邦学习、隐私推断)。可以采取通用的协议实现不同模型的隐私计算,具有可以证明的安全性,但是存在计算的通讯开销大的缺点。 &
          较大 &
          低 &
          高 &
          高 \\ \midrule
        拆分学习       & 按照各方拥有的数据情况,将拆成多份部署在各方,各方交换中间结果完成模型的训练和推断。适用于纵向联邦学习、隐私推断等场景。实现简单,效率较高,但是会一定程度暴露数据和模型的隐私。 & 较大   & 较高  & 较低   & 高    \\ \midrule
        联邦学习(横向)   & 多方共同训练模型,然后在中心化服务器上聚合模型。适用于数据横向分割的情况(各方拥有特征列一致的样本)。只适用于数据横向分割的模型训练场景,且数据隐私存在泄漏风险。 & 较小   & 中  & 中   & 中    \\ \midrule
        可信执行环境     & 在可信硬件上执行中心化的安全计算,基于硬件保证安全性。可适用于多种场景,由于硬件所限,计算效率有一定程度的损失。                                 & 较大   & 中  & 中   & 高    \\ \midrule
        差分隐私 &
          通过在模型计算的特定环节加入精心计算的噪声,使得计算结果实现一定的差分隐私。由于其具体实现仅仅为加入噪声,因此效率损失很小。但是需要在模型精度和安全性之间进行权衡,且其实际的隐私保护程度难以计算,因为差分隐私本身定义来自于“最坏情况的隐私泄露上界”。 &
          较大 &
          高 &
          \multicolumn{2}{c}{相互权衡} \\ \midrule
        密码学和非密码学混合 & 通过损失部分安全性的情况下,对基于密码学的隐私计算协议的效率进行优化。一般可以采用与拆分学习、可信执行环境结合的方式。                              & 较大   & 中  & 中   & 高    \\ \bottomrule
        \end{tabular}
\end{table}

\section{基于拆分学习的隐私保护机器学习}
%
拆分学习~\cite{vepakomma2018split,poirot2019split}是纵向联邦学习(Vertical Federated Learning)~\cite{liu2024vertical}的主流范式,指的是在纵向联邦学习中,把模型拆分成多个部分划分给不同的参与方,参与方之间相互交换中间结果和梯度,从而实现模型的正向传播和反向传播。
%
在此过程中,各方的输入数据并未直接泄露,因此拆分学习被认为在一定程度上可以保护隐私。

\begin{figure}[h]
    \centering
    \includegraphics[width=0.8\linewidth]{Z_Resources/拆分学习示意图.png}
    \caption{拆分学习示意图}
    \label{fig:related_work:split_learning}
\end{figure}

一个简单的拆分学习例子如\autoref{fig:related_work:split_learning}所示。
%
此时纵向联邦学习有3个参与方(记作$P_1, P_2, P_3$),分别拥有样本特征第一部分($\bvec x_1$)、样本特征第二部分($\bvec x_2$)和样本标签($\hat y$)。
%
相应地,模型也被划分成了3个部分,分别是底部模型第一部分($M_{b1}$,参数记为$\Theta_{b1}$),底部模型第二部分($M_{b2}$,参数记为$\Theta_{b2}$),以及顶部模型($M_t$,参数记为$\Theta_h$)。
%
在前向传播的过程中,$P_1$和$P_2$分别本地计算得到各自的中间结果:
$
    \left(
        \bvec h_1 = M_{b1}(\bvec x_1;\Theta_1),
        \bvec h_2 = M_{b2}(\bvec x_2;\Theta_2)
    \right)
$,
然后将其发送给$P_3$(标签拥有方)。
%
$P_3$接收到$\bvec h_1, \bvec h_2$之后,将其输入到顶部模型,计算得到模型输出$y = M_t(\bvec h_1, \bvec h_2; \Theta_h)$。
%
这样便完成了一次前向传播过程。
%
%
在反向传播过程中,$P_3$通过模型的预测标签($y$)和真实标签($\hat y$),计算出损失函数$L(y, \hat y)$,并且将其对于顶部模型和中间结果的参数求梯度得到${\partial L(\hat y, y)}/{\partial \Theta_t}, {\partial L(\hat y, y)}/{\partial \bvec h_1}$ 和 ${\partial L(\hat y, y)}/{\partial \bvec h_2}$.
%
其中,前者被标签拥有方用于更新顶部模型参数,而后两者则发给相应的特征拥有方用于计算损失函数相对于底部模型参数的梯度。
%
该计算通过链式求导法则进行,以$P_1$为例:
\begin{equation}
    \dfrac{\partial L(\hat y, y)}{\partial \Theta_{b1}} = \dfrac{\partial L(\hat y, y)}{\partial \bvec h_1}\cdot \dfrac{\partial \bvec h_1}{\partial \Theta_{b1}}.
\end{equation}
%
注意,上式中的${\partial \bvec h_1}/{\partial \Theta_{b1}}$是雅可比矩阵,链式法则中的乘法也是对应的矩阵乘法运算。

拆分学习具有原理简单、实现方便、计算和通信开销小、可以适用于不同的深度学习模型等优势,因此其已经被应用于很多需要考虑数据隐私的领域,如:医疗影像分割\cite{roth2022split_unet}、边缘设备中的图像分类\cite{fagbohungbe2022split_edge_image,palanisamy2021spliteasy}、毫米波接收功率预测\cite{koda2020split_mmwave}等。

\subsection{拆分学习中的隐私问题}
虽然拆分学习具有实现简单、效率高等诸多优势,但是在训练和推断过程中,各方直接交换了中间结果和中间梯度的明文,从而存在一定的隐私泄漏风险。
%
拆分学习的隐私泄露可以分为两类:特征拥有方的隐私泄露和标签拥有方的隐私泄露。
%
下面将对这两种隐私泄露分别进行介绍。
%
\subsubsection{特征拥有方的隐私泄露}
在拆分学习的过程中,特征拥有方将底部模型产生的中间结果,也就是神经网络的隐层表征(Hidden Representation),直接发送给了标签拥有方。
%
作为输入特征经过变换的结果,隐层表征必然包含了输入特征的信息。
%
假设攻击者腐化了标签拥有方,便可以通过各种手段从隐层表征中恢复出部分输入特征的信息。
%https://www.overleaf.com/project/6564295824afdb06585a588e#
因此,近年来也有许多工作研究了拆分学习中特征拥有方的隐私泄露问题,现总结如下:
%
\begin{enumerate}
    \item 在卷积神经网络中,隐层表征直接包含了输入的信息\cite{abuadbba2020can_split}。
    由于卷积层的局域运算特性,其会保留图像本身的大致形状、轮廓等信息。
    即便经过多个卷积层后,隐层表征依然能够保留图像的整体轮廓,从而使得原始图像的数据出现一定程度的泄露。
    %
    \item 攻击者可能训练一个逆向网络从隐层表征中重构出原始输入特征~\cite{vepakomma2020nopeek,hezecheng_2019_model_inversion_attack}。
    一种简单的情况可以是,攻击者能够获取一部分泄漏的样本特征$(\bvec x_1, \cdots, \bvec x_n)$和对应的隐层表征$(\bvec h_1, \cdots, \bvec h_n)$,则最优的重构函数$R_*$可以定义为:
    \begin{equation}
        R_* = \mathop{\text{argmin}}_{R} \sum_{i=1}^n \Vert R(\bvec h_i) - \bvec x_i\Vert^2 = 
        \mathop{\text{argmin}}_{R} \sum_{i=1}^n \Vert R(M_b(\bvec x_i)) - \bvec x_i\Vert^2,
    \end{equation}
    其中,$\Vert \cdot\Vert^2$ 表示二范数平方,也可以按照具体情况换成其他的函数来度量重构效果。
    一般可以采用一个和底部模型相对应的神经网络作为重构函数,比如原始模型是是多层卷积,则$R$可以主要由反卷积(ConvTranspose)构成。
    %
    \item 如果攻击者本身拥有底部模型的参数信息,则可以采用白盒攻击的形式,优化$\bvec x'$使得$M_b(\bvec x')$尽可能接近$\bvec h = M_b(\bvec x)$~\cite{hezecheng_2019_model_inversion_attack,luoxinjian2021feature_attack}:
    \begin{equation}
    \label{eq:related_work:whitebox-reconstruction}
        \bvec x'_* = \mathop{\text{argmin}}_{\bvec x'} \Vert M_b(\bvec x') - \bvec h\Vert^2 = \mathop{\text{argmin}}_{\bvec x'} \Vert M_b(\bvec x') - M_b(\bvec x)\Vert^2.
    \end{equation}
    %
    注意到\autoref{eq:related_work:whitebox-reconstruction}可能有无穷多个解,影响重构效果。为了提高重构效果,也可以在损失函数中加入关于输入特征的先验知识。
    %
    比如输入特征是图像,则可以在损失函数中加入$\bvec x'$的总变差(Total Variance),使得重构出来的结果更接近实际图像;
    如果攻击者有同一样本的额外特征$\bvec z$,则可以将$\bvec x'$替换为$g(\bvec z)$,这里假设想要攻击的输入特征和攻击者获取的额外特征有关。
    %
    \item 在训练过程中,攻击者可以对训练目标进行修改,使得中间结果包含更多关于输入特征的信息。比如假设攻击者拥有一部分泄漏的样本特征$\{ \bvec x' \}$,
    则他可以先训练一个自编码器~\cite{kramer1991autoencoder,baldi2012autoencoders}$(f,f^{-1})$使得$f^{-1}\circ f(\bvec x')\approx \bvec x'$。
    %
    在拆分学习训练过程中,攻击者采用对抗训练的方式,使得待攻击样本的隐层表征$\bvec h = M_b(\bvec x)$与自编码器编码的泄漏样本的特征$f(\bvec x')$尽可能接近。
    %
    于是攻击者可以直接通过$f^{-1}(\bvec h)$来重构原始的输入特征~\cite{pasquini2921inference_attack_tiger}。
    %
\end{enumerate}

值得注意的是,第4种方法需要攻击者改变正常的拆分学习训练过程,而前面介绍的3种方法则不需要改变训练过程。
%
一般将主动改变训练过程的攻击者称为“主动攻击者(Active Attacker)”,反之则称为“被动攻击者(Passive Attacker)”。
%
主动攻击者往往能取得更强大的攻击效果,但是由于其对训练本身的改变,可能导致训练效果变差、训练速度变慢等问题,会更容易被检测出来,且无法在模型推断阶段进行攻击。
与此相反,被动攻击者的攻击效果会更低,但是具有更好的隐蔽性。

\subsubsection{标签拥有方的隐私泄露}
对于标签拥有方的隐私泄露,我们考虑的是攻击者腐化特征拥有方的情况。
%
标签拥有方的隐私泄露主要来源于两个方面:
\begin{enumerate}
    \item 隐层表征(中间结果)导致的标签信息泄露。
    深度学习模型学习的过程,可以看做一个将输入特征逐渐转化成标签的过程。
    因此,随着训练的进行,隐层表征会逐渐变得与标签更为相关。
    %
    许多将神经网络隐层表征可视化的工作也显示出,随着训练的进行,隐层表征逐渐按照对应的标签聚类成不同的簇;且越靠近输出的隐层表征和标签关联度越大~\cite{paulo2017visualize_hidden,pezzotti2017deepeyes,cantareira2020hidden_vector_fields}。
    %
    在这种情况下,攻击者就可以根据隐层表征来推断输入数据的标签。
    %
    即使在没有任何额外信息的情况下,攻击者也可以对隐层表征进行聚类,从而获取输入样本的类别关系~\cite{liujunlin2022clustering_attack,liujunlin2023distance_attack}。
    %
    如果攻击者能够获取少量泄漏的隐层表征和对应的标签,也可以自己训练一个顶部模型,并且获得很好的分类效果~\cite{fucong2022label_infer_attack}。
    %
    \item 隐层表征的梯度导致的标签信息泄露。
    与隐层表征类似,隐层的梯度和标签信息也呈现出非常强的相关性。
    %
    比如在二分类模型中,同类样本的梯度间的余弦相似度接近1,而异类样本的梯度间的余弦相似度则接近-1,直接暴露了样本的类别信息~\cite{oscarli2022label_defense_marvell}。
    %
    此外,也可以通过优化替代样本的梯度,使其尽可能接近攻击者获取到的梯度,从而复原出样本的标签~\cite{erdogan2022unsplit}。
    %
    具体方法如下:
    \begin{equation}
        \mathop{\text{argmin}}_{\Theta_t', \hat y'} \left\Vert \dfrac{\partial L[\hat y', M_t(M_b(\bvec x);\Theta_t')]}{\partial M_b(\bvec x)} - \dfrac{\partial L[\hat y, M_t(M_b(\bvec x);\Theta_t)]}{\partial M_b(\bvec x)} \right\Vert^2,
    \end{equation}
    其中,$\bvec x, \hat y$ 是原样本的特征和标签,而$\hat y'$是随机初始化的替代标签;$\Theta_t, \Theta_t'$分别表示原始的顶部模型参数和替代的顶部模型参数。
    通过优化替代的样本和顶部模型参数,可以恢复出样本的标签。
\end{enumerate}

标签泄露不仅仅带来了样本标签的隐私问题,同时还会泄露模型本身。
%
如上文所述,当特征拥有方腐化(Corrupted)后,它可以利用少量标签(或直接聚类生成标签),训练出顶部模型。
%
由于在模型训练的后期,底部模型的表征提取能力已经很强,固定底部模型后,攻击者很容易就可以训练出一个表现良好的顶部模型,从而得到整个完整的模型。
%
因此,这种攻击也被称为“模型补全攻击(Model Completion Attack)”~\cite{fucong2022label_infer_attack}。
%
在这种情况下,腐化的特征拥有方可以窃取几乎标签拥有方的所有资产(标签和顶部模型),从而使得拆分学习的隐私保护能力受到严重损害。


\subsubsection{总结}
%
无论是输入特征的泄露还是标签的泄露,都与许多因素有关,包括模型结构和拆分的层数。
%
显而易见的是,如果分割点靠近模型输入,则隐层表征泄漏的特征信息更多;而分割点接近模型输出,则会泄露更多关于标签的信息。
%
这也是著名的数据处理不等式(Data Processing Inequality)的推论。
%
模型的结构也对隐私泄漏的程度有很大的影响。
%
从输入特征的角度而言,
%
对于卷积神经网络,卷积层的隐层表征与输入十分相关,且很多网络的隐层维度也很高,因此很容易被攻击者重构出输入特征~\cite{abuadbba2020can_split};
%
对于全连接网络,如果隐层维度小于数据维度,则会存在无穷多组输入拥有同样的隐层表征,
这导致没有先验知识的情况下,攻击者难以获取输入特征信息;
%
基于Transformer结构的语言模型,包括时下流行的大语言模型(Large Language Model),拥有很高的隐层表征维度,并且保留了序列结构,因此攻击者也可以轻易地从隐层表征中重构出原始的输入文本~\cite{morris2023embedding_almost}。
%
从标签的角度而言,更低的隐层表征维度往往意味着更容易恢复出标签,因为高维度的特征中可能包含许多与标签无关的信息,使得攻击者更难从中提取出标签~\cite{oscarli2022label_defense_marvell,sunjiankai2022forward_embedding_protect}。


\subsection{拆分学习的隐私保护方法}
为了解决前文所述的拆分学习中的隐私泄露问题,一些研究也提出了保护拆分学习中输入特征和标签的隐私的方法,当前主要采用的方法是对隐层表征或隐层梯度进行扰动。

首先介绍对隐层表征进行扰动,使其与输入特征或标签特征不相关的方法。
%
由于隐层表征、输入特征、标签的维度可以是任意的,因此常用的皮尔逊相关性(Pearson Correlation)并不适用。
%
因此现有工作主要采用距离相关性来衡量隐层表征和输入特征、标签的相关性~\cite{vepakomma2020nopeek,sunjiankai2022forward_embedding_protect}。
%
距离相关性(Distance Correlation)~\cite{szekely2007dcor,szekely2009brownian_dcor}是一种特殊的相关性度量,可以度量不同维度的随机向量之间的相关性,并且可以度量非线性的相关性,距离相关性为0当且仅当两个变量是无关的(互信息为0)。
%
使用距离相关性来保护输入特征或标签信息的损失函数可以写为:
\begin{equation}
    L' = L_0 + \alpha \text{Dcor}(H, X) + \beta \text{Dcor}(H, Y),
\end{equation}
其中,$\text{Dcor}(\cdot, \cdot)$表示距离相关性,$X,H,Y$分别表示当前批次的输入特征、拆分层表征和标签,$\alpha,\beta$则用于控制扰动程度。
%
注意到,此处的距离相关性计算是通过批样本的经验分布(Empirical Distribution)进行估计的,因此需要较大的批大小(Batch Size),否则估计值会出现较大的误差。
%

针对隐层梯度泄露标签信息的情况,Marvell方法~\cite{oscarli2022label_defense_marvell}通过最小化正样本和负样本之间的隐层梯度间的KL散度来保护样本的标签信息。
%
当标签拥有方计算出隐层梯度$g = \partial L/\partial \bvec h$时,它会对其加入一个正态分布的噪声$D \sim \mathcal N(0, \Sigma_D)$,然后将其发给特征拥有方。
%
具体的优化问题可以写做:
\begin{equation}
\begin{split}
\label{eq:related_work:marvell}
    & \min_{\Sigma_D^+,\Sigma_D^-} \text{KL}\left[\mathcal N(\mu_g^+, \Sigma_g^+ + \Sigma_D^+)\Vert\mathcal N(\mu_g^-, \Sigma_g^- + \Sigma_D^-)\right],
    \\
    & \text{s.t.}\quad p\text{tr}(\Sigma_D^+) + (1-p)\text{tr}(\Sigma_D^-) \le P,
\end{split}
\end{equation}
这里假设正样本和负样本的原始的隐层梯度也分别满足正态分布$\mathcal N(\mu_g^+, \Sigma_g^+)$和$\mathcal N(\mu_g^-, \Sigma_g^-)$,
而$\Sigma_D^+,\Sigma_D^-$则分别是给正样本和负样本的隐层梯度家的噪声的协方差矩阵。
%
约束条件中的$p$和$1-p$分别代表正样本和负样本的频率,$P$表示设定的噪声大小的平均值上限,用于控制噪声的规模,防止添加的噪声太大。
%


上述对隐层表征和梯度的扰动方法均存在一些缺陷。
%
在隐层表征维度高的情况下,距离相关性的估计会变差,从而使得其保护效果大幅度降低~\cite{erdogan2022unsplit}。
%
而Marvell方法~\cite{sunjiankai2022forward_embedding_protect}虽然对梯度进行了扰动,并不能防止从隐层表征直接推出样本标签的情形,
同时该方法也仅适用于二分类场景。
%
因此,如何保护拆分学习过程中的特征和标签隐私,依然是个极具挑战性的问题。


\subsection{拆分学习的效率提升}
相对于基于密码学方法的隐私保护机器学习,拆分学习具有较小的开销,因为拆分学习仅仅是在明文计算的基础上,传输隐层表征和隐层梯度,并未引入额外的计算。
%
但是考虑到许多神经网络的隐层尺寸较大,因此在通信效率方面,拆分学习依然存在可以优化的空间。
%
一种简单的思路是将稀疏(Sparsification)和量化(Quantization)方法引入拆分学习,对隐层表征和隐层梯度进行压缩,从而减少拆分学习训练和推断过程中的通信量。
%

\textbf{稀疏化:}
对一个向量(或矩阵、张量)进行稀疏化,指的是将其(绝对值)较大的元素保留,而将其绝对值较小的元素丢弃,即设置为0。
%
稀疏化有效性基于“较大的元素在计算过程中较为重要,而较小的元素在计算过程中可以忽略”这一假设。
%
稀疏化之后,数据的传输格式也会发生变化。一种简单的做法是将数据以“(下标,值)”的形式进行编码。假设原有的数据位数为$L$(如对于32位浮点数$L=32$),数据的总量为$N$,稀疏率为$p$,则简单的下标-值编码方法可以实现压缩比率(压缩后大小/压缩前大小)为:
\begin{equation}
    \text{压缩比率}=\dfrac{pN(L + \lceil \log_2N \rceil)}{NL} = p(1 + \dfrac{\lceil \log_2N \rceil}{L}).
\end{equation}
为了提高压缩比率,可以采用更加先进的数据压缩方法,如经典的Huffman编码~\cite{huffman1952}可以直接应用于稀疏化后的数据(的二进制表示)上。
%
更为适合稀疏化的编码为对下标的差值(Run-length)序列进行Golomb编码~\cite{gallager1975golomb}。
%
如果在一批数据中,每个下标被稀疏化的概率是均等的,则Golomb编码具有最小的期望压缩比率,大约可以在下标-值编码的基础上再减少一半~\cite{sattler2019sparse_binary}。
%
值得注意的是,采用编码方案虽然减少了通信开销,但是也会带来一定的计算开销。


\textbf{量化:}
量化指的是降低数据的精确度,使用较少的比特位保留数据,在尽可能保证数据的准确性的情况下,压缩其存储空间。
%
一般的机器学习模型的训练或推断采用的都是32位浮点数,通过量化的方式,可以将模型的权重、模型计算的中间结果等压缩到更低位数(如8位,4位,甚至1位),从而减少模型大小或降低模型计算过程中的内存开销和计算开销~\cite{zhou2016dorefa,banner2018_8bit,yang2019quantization}。




利用稀疏化和量化减轻通信量的方法,在分布式计算、横向联邦学习领域被广泛应用~\cite{wen2017terngrad,sattler2019sparse_binary}。
%
此时,稀疏化和量化可以同时被应用在表征和梯度上,从而减少了每轮训练过程中的通信量开销。
%
虽然稀疏化和量化导致梯度不准确,由于使用随机梯度下降(Stochastic Gradient Descent)优化时,批样本梯度本身有一定噪声,甚至被认为对收敛(Convergence)和泛化(Generalization)是有益的~\cite{hardt2016sgd,goyal2017sgd_imagenet,chaudhari2018sgd}。
%
同时,相关研究也表明,稀疏化和量化并未对模型最终的结果有显著影响,而训练到一个较高准确度的通信量也显著小于常规训练的通信量~\cite{aji2017sparse,sattler2019sparse_binary,wen2017terngrad}。

%
但是拆分学习作为一个较新的领域,其通信效率提升的研究相对较少。
%
Castiglia等人~\cite{castiglia2022compressed_vfl}研究了基本的Top-$k$稀疏化、(标量)量化以及向量量化(Vector Quantization)在拆分学习中的应用,并且证明了其收敛性。
%
具体而言,拆分学习训练时的前向传播过程被更改为
\begin{equation}
\label{eq:split-compress}
    Y = M_t(\mathsf{Compress}[M_b(X)]),
\end{equation}
%
其中,$M_t$表示顶部模型,$M_b$表示底部模型,$X, Y$ 表示输入特征和模型预测值,$\mathsf{Compress}$表示压缩操作(如稀疏、量化)。
%
尽管收敛性得到证明,但是该文采用的压缩方法最多只能将压缩率降低到1/16,且常规的稀疏和量化(除了向量量化外)对模型最终的效果有较明显的降低。
%
此外,该方法只适用于训练场景,并未考虑推断场景,并且要求公开的顶部模型让各个参与方执行块坐标下降法(Block Coordinate Descent)对自身的参数进行优化。

也有部分研究采用异步更新的方式提高拆分学习的通信效率,其方法包括底部模型多轮本地更新~\cite{fu2022cache_vfl}或是顶部模型多轮本地更新~\cite{chen2021async_split}。
%
这些方法同样针对的是拆分学习训练的场景,并不能应用于推断阶段。
%
此外,Ayad等人~\cite{ayad202vfl}提出使用自编码器压缩中间表征,也就是将\eqref{eq:split-compress}中的$\mathsf{Compress}$替换为一个自编码器,从而实现训练和推断过程中的通信优化。
%
但是该方法需要针对特定模型定制自编码器,并非一种通用的方法。
\section{基于密码学的隐私保护机器学习}
纵向联邦学习可以看作是特定的一类隐私计算问题,即:在保护数据和模型参数的情况下进行模型的训练和推断,因此也可以通过基于密码学的安全多方计算(Secure Multiparty Computation)来实现。

\begin{definition}[安全多方计算]
    有$n$个参与方$P_1, \cdots, P_n$与各自的输入$X_1, \cdots, X_n$,以及一个公共函数$f$,安全多方计算指的是各方根据指定的协议交互计算出$Y= f(X_1, \cdots, X_n)$($Y$可以是公开的或是被指定的参与方获得,两种定义是等价的),同时保护各方输入的隐私。
\end{definition}

注意安全多方计算的定义隐含了安全性设定(Security Setting),也就是对于各个参与方行为的限制条件。
安全性设定主要可以分为两种,包括:
各个参与方忠实按照协议规则执行协议但是同时利用自己获取到的一切信息,称为半可信(Semi-honest)安全性设定;
以及参与方可能存在不遵守计算协议的情况,称为恶意(Malicious)安全性设定;
此外,也有在上述两种情况下存在部分参与方共谋(Collusion)的设定。
%
复杂的安全性设定可能导致极为复杂的协议设计和安全性证明。
%
本文遵循一般隐私保护机器学习中的半可信安全性设定~\cite{wagh2019securenn,mohassel2018aby3,riazi_2018_chameleon}。
%
%


安全多方计算协议往往基于几种特定的密码学原语来实现,包括秘密分享(Secret Sharing)、同态加密(Homomorphic Encryption)、混淆电路(Garbled Circuits)等。
%
下面对此进行简要介绍。
%


\textbf{秘密分享}:
秘密分享~\cite{shamir1979share}指的是把一个值分享给多个参与方,其中一定数量的参与方合作才能恢复出该值。
%
具体而言,
%
$(t,n)$-秘密分享指的是将一个值$x$分享给$n$方$P_0, \cdots, P_{n-1}$,其中$P_i$得到的值记作$\langle x \rangle_i$。
%
至少要$t$个参与方一起,才能恢复出$x$的值。低于$t$个参与方则无法得到任何关于$x$的信息。
%
常用的秘密分享包括两方加法分享(Additive Sharing)以及两方布尔分享(Boolean Sharing),这两种分享分别把一个数$x$拆分成两个随机数相加($x = \langle x \rangle_0 + \langle x \rangle_1$)或是两个随机数(按位)异或($x = \langle x \rangle_0 \oplus \langle x \rangle_1$)。
%
在秘密分享的情况下,进行加法(异或)只需各方本地运算,但是进行乘法(异与)较为复杂,一般可以采用离线(Offline)计算的Beaver三元组进行~\cite{beaver1992efficient},即:在获得具体的输入前,两方通过一定方法计算出一组秘密分享的随机数$(u,v,w)$满足$uv = w$。
%
一般两方在离线阶段可以使用同态加密~\cite{paillier1999,gentry2009fully}、不经意传输(Oblivious Transfer)~\cite{yadav_2022_ot_survey}等手段生成Beaver三元组;
在三方场景下,可以使用一个半可信第三方生成分发从而提高效率。
%
注意秘密分享一般在有限域(Finite Domain)上进行,如:加法分享在模$N$的整数环$\mathbb Z_N$进行,布尔分享按照定义在$\mathbb Z_2$进行)。
%
此时各方所获取到的值在自身视角中都是在有限域上均匀分布的随机数,与实际值无关,因此实现信息论安全(Information-Theoretic Security)。


\textbf{同态加密}:
同态加密是一类特殊的加密算法,其支持在密文上进行一定的计算,使得解密的结果和明文计算一致。
%
令某种加密系统的加密函数为\textsf{Enc},解密函数为\textsf{Dec}。
%
如果对于某个定义在明文上的运算符$\mathsf{OP}_P$,存在密文上的运算符$\mathsf{OP}_C$,使得对于任意明文$X, Y$都满足:
\begin{equation}
    \textsf{OP}_P(X, Y) = \mathsf{Dec}[\mathsf{OP}_C (\mathsf{Enc}[X], \mathsf{Enc}[Y])],
\end{equation}
%
则称该加密系统可以同态地计算$\mathsf{OP}_P$的运算。
%
注意$\mathsf{OP}_P$ 往往不等于 $\mathsf{OP}_C$。
比如,Paillier同态加密支持明文加法,即:$\mathsf{OP}_P=$`$+$',但是对应的密文运算是乘法,也就是$\mathsf{OP}_C=$`$\times$'。
%
同态加密按照支持的运算的不同,一般可以分为半同态加密(Partial Homomorphic Encryption)和全同态加密(Fully Homomorphic Encryption)。
%
半同态加密支持一种同态运算,如加法或乘法,分别可以称为加同态加密和乘同态加密。
%
全同态加密同时支持加法或乘法。
%
除此之外,还有分级(Leveled)全同态加密,其只支持一定次数多乘法运算。
%
当前常用的同态加密包括Paillier加同态加密~\cite{paillier1999},其定义域为大整数;BFV/BGV全同态加密~\cite{2012bfv1,2012bfv2,2014bgv},其定义域为整系数多项式;以及CKKS全同态加密~\cite{ckks2017},其定义域是实系数多项式,并且运算过程会产生一定的误差。


\textbf{混淆电路}:
混淆电路由姚期智院士于1986年提出~\cite{yao1986gc},是一个两方安全计算协议,可以用于计算任意布尔电路。
%
混淆电路通过对原始电路的逻辑门进行“混淆”得到,具体而言,每个门的输入和输出都变成了随机数,因此无法根据混淆后的电路的输入、输出以及中间值,获取关于原始电路输入输出值的任何信息。
%
混淆电路的执行过程可以表示为:给定一个公开函数$f$,某方(这里设为$P_0$)产生一个布尔电路$C$,然后对其进行混淆得到混淆电路$C^G$,并发送给另一方(这里设为$P_1$);同时,$P_0$自身的输入$X_0$ 混淆后得到 $X_0^G$
%
同时,$P_1$和$P_0$ 执行不经意传输协议~\cite{yadav_2022_ot_survey},获取$P_1$输入的混淆值 $X_1^G$。
%
然后$P_1$可以根据混淆输入计算出混淆输出$Y_G = C_G(X_0^G, X_1^G)$,将其返还给$P_0$后,$P_0$根据自身维护的混淆表得到实际的输出值。
%
混淆电路只需要常数轮的通信即可计算任何门电路的值,但是因为电路中的每个布尔值都转化成了高位的随机数,会消耗大量通信流量;由于涉及密码学计算,也有较大的计算开销。
%
混淆电路发展至今有多种优化方案提出,包括Free-XOR~\cite{kolesnikov2008free_xor},HalfGate~\cite{zahur2015halfgate}等,降低了通信和计算开销。
%

除去以上介绍的几种密码学技术,多方安全计算还可以包含其他技术,以及一些定制化的计算协议。
如不经意传输~\cite{yadav_2022_ot_survey,chou_2015_simplest_ot}可以使得一方从另一方拥有的多个值中按照下标选取一个值,同时不暴露查询的下标以及未被选择的值。
%
下文将对部分主要的基于密码学的隐私保护机器学习研究进行介绍。


\subsection{基于单一密码学技术的隐私保护机器学习}
本节对采用单一密码学技术来实现隐私保护机器学习的相关工作进行介绍。
%
这些工作基于两方的隐私保护机器学习场景,如MLaaS(Machine Learning as a Service),即:用户提供数据,服务方提供机器学习模型,两方进行训练或推断的情况。
%

\textbf{基于同态加密}:
2016年微软提出CryptoNets~\cite{gilad2016cryptonets},率先通过同态加密实现神经网络的推断。
%
由于神经网络的权重和输入都被加密,因此这种方法可以实现安全的神经网络推断。
%
CryptoNets采用了YASHE加密算法~\cite{bos2013yashe},这是一种分级同态加密算法,可以支持一定次数的乘法运算。
%
由于同态加密仅支持加法和乘法运算,因此CryptoNets将神经网络中的非线性激活函数替换成了平方函数,然后在明文上训练得到模型权重。
%
该方法需要花费数分钟和300Mb左右的流量对MNIST数据集中的一张图片进行分类。
%
%
一些研究~\cite{hesamifard2017cryptodl,chabanne2017pp_dnn}对CryptoNets进行了改进,包括使用更加精确的多项式拟合激活函数、将同态加密算法换为BGV~\cite{2014bgv}或CKKS~\cite{ckks2017}等更高效的算法,以及优化同态加密的密文打包技术等。
%
Zhou等人~\cite{zhoujunwei2020binary_encrypted_nn}通过将卷积神经网络的权重二值化,实现了更高效的基于全同态加密的神经网络推理。

\textbf{基于混淆电路}:
DeepSecure框架~\cite{rouhani2018deepsecure}将神经网络模型转化为布尔电路,从而实现隐私神经网络推断。
%
为了提高效率,该框架减少了各类电路中的非异或门数目,从而适配Free-XOR算法。
%
该框架运算速度高于CryptoNets,但是带来了更高的通信开销。
%
实验表明,一个简单的全连接网络的推断消耗800MB左右的流量。
%
XONN框架~\cite{riazi2019xonn}将二值神经网络(Binary Neural Network)~\cite{qinhaotong_2020_binary_nn}与混淆电路结合,将内积运算转变为异或运算,从而极大地提高了安全推断的效率。


\subsection{混合多种密码学技术的隐私保护机器学习}
为了提高隐私保护机器学习的效率,现有的隐私保护机器学习框架大多将多种密码学技术混合,为机器学习中不同的算子找到最佳的密码学技术。
%
我们在下文将主流的基于密码学的隐私保护机器学习框架按照参与方个数分别介绍。

\textbf{两方框架}:
ABY框架~\cite{demmler2015aby}提出将算术秘密分享、布尔秘密分享以及YAO分享(基于混淆电路~\cite{yao1986gc}的两方分享,一方持有混淆值和真实值的对照表,另一方持有混淆电路和混淆值)融合,并且设计协议对三种分享状态的值进行相互转化,从而为特定的计算任务找到最适合的混合计算协议。
%
SecureML框架~\cite{mohassel2017secureml}将算术秘密分享和混淆电路结合,实现了逻辑回归的推断和训练。
%
其还使用分级同态加密和不经意传输对算术秘密分享的Beaver三元组~\cite{beaver1992efficient}的离线生成进行了优化,并且将Sigmoid函数表示为分段线性函数,通过混淆电路(计算当前值落在哪一段)和算术秘密分享融合来提高计算效率。
%
MiniONN框架~\cite{liujian2017minionn}将秘密分享和混淆电路进一步应用于神经网络推断中,并且使用分段的Spline插值来拟合激活函数,以及采用SIMD(Single Instruction Multiple Data)技术加速了Beaver三元组的生成,从而提高了隐私神经网络的模型准确率和效率。
%
Gazelle框架~\cite{juvekar2018gazelle}使用优化的基于格密码的同态加密(BFV加密~\cite{2012bfv1,2012bfv2})直接加速安全矩阵乘法计算,而非用于产生三元组,从而极大降低了通信开销;同时采用混淆电路来计算ReLU、最大池化(Max Pooling)等非线性运算。
实验结果表明Gazelle的性能甚至大幅度优于需要依赖第三方的Chameleon框架。
%
Delphi框架~\cite{mishra2020delphi}基于Gazelle框架进行了改进,将同态加密的安全乘法计算用于离线计算Beaver三元组,并利用了神经网络推断过程中权重不变的特性减少了在线的开销;同时将ReLU激活函数替换成二次函数以降低开销。
%
CryptFlow2框架~\cite{rathee2020cryptflow2}基于秘密分享,采用同态加密和不经意传输实现安全乘法,并且新设计了高效的基于不经意传输的安全比较和除法协议,将两方的隐私保护机器学习推广到ResNet~\cite{hekaiming2016resnet}等大型模型中。
%
Cheetah框架~\cite{huang2022cheetah}进一步对基于同态加密的神经网络线性运算以及基于不经意传输的ReLU激活函数等非线性运算进行优化。
%si
SIRNN框架~\cite{rathee2021sirnn}通过查真值表的方式计算指数函数,进一步对神经网络中的Softmax等非线性函数进行了优化。

\textbf{三方框架}:
Chameleon框架~\cite{riazi_2018_chameleon}混合秘密分享、GMW协议~\cite{gmw_1987}和混淆电路协议进行神经网络计算,在离线阶段使用了可信硬件~\cite{sabt_2015_tee}作为第三方,并采用相关随机性(Correlated Randomness)在各方生成相同的随机数,提高了Beaver三元组产生的效率。
%
ABY3框架~\cite{mohassel2018aby3}基于ABY框架~\cite{demmler2015aby},将秘密分享、布尔分享和混淆电路推广到三方计算的场景以提高效率,并对乘法截断进行了改进。
%
SecureNN框架~\cite{wagh2019securenn}采用了两方秘密分享并且使用第三方来产生Beaver三元组,同时基于比较的布尔运算电路设计了高效的三方安全比较协议,实现了高效的隐私神经网络推断。
%
CryptFlow~\cite{kumar2020cryptflow}是一个支持两方和三方协议的开源框架,其两方协议采用的是ABY协议,而三方协议基于SecureNN框架,对其卷积操作的Beaver三元组进行了优化,同时使用相关随机性来进一步降低通信量;该框架还提供了基于可信执行环境(Trusted Execution Environment)~\cite{sabt_2015_tee}的恶意安全转换机制,可以将半可信安全的协议转换为恶意安全的协议。
%
Crypten框架~\cite{knott2021crypten}基于秘密分享和GMW协议~\cite{gmw_1987}实现安全的深度学习,并且提供了类似Pytorch~\cite{2019_pytorch}风格的接口,使得非密码学专业研究者也可以进行隐私保护机器学习。
%
Falcon框架~\cite{wagh2021falcon}基于ABY3和SecureNN进行了改进,优化了比较协议。
%
AriaNN框架~\cite{ryffel2021ariann}提出使用函数秘密分享(Function Secret Sharing)来计算比较函数,将其降低到仅需一轮通信轮次。

\subsection{其他相关研究}
除了设计安全多方计算框架之外,也有一些其他关于提高基于密码学的隐私保护机器学习效率的研究。
%
比如,
Dalskov等~\cite{dalskov2020secure_q8}研究了将量化(Quantization)技术应用于隐私保护机器学习,将神经网络量化到8比特,并在多个现有的隐私保护机器学习框架上进行了测试。
%
此外,也有许多对于非神经网络的专用机器学习模型进行隐私保护的研究,如树模型~\cite{wu2020vf_tree,fang2021secure_xgb,lu2023squirrel}、聚类模型~\cite{bunn2007secure_kmeans,wu2020secure_kmeans}、矩阵分解~\cite{nikolaenko2013ppmf,kim2018ppmf}等。
%
这些方法往往基于已有的同态加密、混淆电路、多方安全计算协议等技术,对于特定的机器学习模型进行设计和优化。
%


2023年起,随着ChatGPT~\cite{chatgpt}的出现,大语言模型在工业界和学术界产生了巨大的影响。
%
而大模型的参数量和计算量都极为巨大,即使是明文推断的情况下也对硬件有较高要求,因此对基于密码学的隐私保护机器学习提出了新的挑战。
%
Iron Transformer~\cite{hao2022iron},MPCFormer~\cite{li2022mpcformer},SecureTLM~\cite{chen2024securetlm},PrivFormer~\cite{akimoto2023privformer}等框架在之前研究的基础上,采用同态加密、秘密分享、多项式拟合ReLU、使用特定模型结构等方法,实现了Transformer模型的隐私保护推断。
%
针对参数量高达数十亿的大语言模型,最近也有研究基于已有的ABY3协议或Cheetah框架进行~\cite{dong2023puma,lu2023bumblebee,hou2023ciphergpt}。
目前这些方案在生成单个输出单词时需要至少进行数十GB的流量传输,在理想环境下也需要数分钟的计算时间,因此依然难以用于实际应用。




\section{密码学与非密码学混合的隐私保护机器学习}
考虑到基于密码学的隐私保护机器学习框架往往开销巨大且实现困难,也有许多研究尝试将密码学和非密码学方法进行融合,从而实现更高效的隐私保护机器学习。
%

一种方法是将同态加密和其他方法结合,其计算流程大致可以表示为
\begin{equation}
    \llbracket \bvec y \rrbracket = W \otimes \llbracket \bvec x \rrbracket  + \llbracket \bvec b \rrbracket,
\end{equation}
其中,$\llbracket \cdot \rrbracket$表示加同态加密的密文,$\otimes$ 表示明文和密文的同态乘法,其结果为乘积的密文。
%
GELU-Net算法~\cite{zhangqiao_2018_gelu_net}利用同态加密实现了两方的安全神经网络推断。
该算法假设用户拥有同态加密的私钥。
对于神经网络的每一层,用户将数据上传至服务器,服务器执行同态乘法运算,得到加密的结果后返回给用户,用户再计算非线性的激活函数值。
%
该方法虽然保护了用户的隐私,但是若用户多次进行查询,即可通过对多个输入$\bvec x_i$和输出$\bvec y_i$ 解线性方程组$(W\bvec x_i + b = y_i)$,恢复出神经网络的权重。
%
BAYHENN算法~\cite{xiepeichen_2019_bayhenn}类似Gelu-Net,区别在于其则采用贝叶斯神经网络来保证模型的隐私。
在贝叶斯神经网络中,模型的权重是随机变量,使得根据输入和输出学习权重转换成一个“噪声学习(Learning With Errors)”问题,而该问题被证明是一个困难问题。
%
然而后续研究指出~\cite{wong_2020_lwe_model},BAYHENN和GELU-Net的安全性都存在问题,很容易被攻击者通过选取特定输入的方式窃取模型参数

此外,还有一些研究通过结合拆分学习和密码学方法,提高了隐私保护机器学习的效率。
Zhou等~\cite{zhou_2022_codesign}提出在多方的纵向联邦学习中使用秘密分享计算神经网络的第一个全连接层,此后在服务器上明文计算。
Chen等~\cite{chen2020vertically}将该方法进一步扩展到图神经网络中。
%
BlindFL框架~\cite{fu2022blindfl}将纵向联邦学习的底部模型拆分层使用密码学方法计算,同时明文计算底部模型的其他层和顶部模型。
%
这些方法在一定程度上权衡了效率和隐私,相比于传统的拆分学习,其一般只暴露多个底部模型产生的联合表征而非单一参与方底部模型的输出。
%
但是由于中间结果的暴露,其依然面临着较大的隐私泄漏风险,存在着被逆向攻击的可能性~\cite{hezecheng_2019_model_inversion_attack,abuadbba2020can_split,luoxinjian2021feature_attack,erdogan2022unsplit,qiupengyu_2023_label_selling_you_out}。

综上所述,当前密码学和非密码学方法混合实现隐私保护机器学习的研究仍然处于起步阶段。
%
